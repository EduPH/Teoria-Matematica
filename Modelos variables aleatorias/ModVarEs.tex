\chapter{Distribuciones especiales}

\section{Distribución chi cuadrado ($\chi $)}

Una $\chi^2_n$ con $n$ grados de libertad es la suma
de $n$ distribuciones $N(0,1)$ al cuadrado. Es decir,
$$\chi_n^2=\sum_{i=1}^nX_i^2,\quad X_i\text{ con } F\in N(0,1),\forall i\in \mathbb{N}$$

\begin{itemize*}
\item $\chi_{n_1}^2+\chi_{n_2}^2=\chi_{n_1+n_2}^2$
\item $E\chi_n^2=n$
\item $\frac{\chi_n^2}{n} \xrightarrow{P} 1$
\item $\frac{1}{\sigma²}\sum(Y_k-\bar{Y})²$ se distribuye según una chi cuadrado
  con $n-1$ grados de libertad. Con cada $Y_i$ con distribución $N(\mu,\sigma²)$
\end{itemize*}

\section{Distribución t de student}
Si tenemos
$$X \text{ una } N(0,1)$$
y
$$Y \text{ una } \chi_n^2$$
entonces
$$\frac{X}{\sqrt{\frac{Y}{n}}} \text{ se distribuye según una } t_n$$

\begin{itemize*}
\item la función de densidad es simétrica.
\item Tiende en ley a una $N(0,1)$
\item $\sqrt{n}\frac{\bar{X}-\mu}{S_c} \text{ se distribuye según una } t_{n-1}$
\end{itemize*}

\section{Distribución F de Snedecor}

Sean

$$X,Y \text{ con distribuciónes } \chi_n^2 \text{ y } \chi_m^2,\text{ independientes.}$$
Entonces se tiene,

$$\frac{\frac{X}{n}}{\frac{Y}{m}} \text{ se distribuye según una } F_{n,m}$$
